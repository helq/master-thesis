\chapter{Related Work}\label{related-work}

{\ichange{Revise, rewrite and extend this subsection}}

A big, widely used, and mature language like Python has no lack of
Static Analysis tools. A big, widely used, and mature area like Machine
Learning has no lack of people trying to build tools to make writing
code for it easier. In this chapter, a brief overview of the many Static
Analysis tools developed for Python code and the several approaches
taken to solve the problem of tensor shapes.

\section{Static Analysis in Python}\label{static-analysis-in-python}

{\inlinetodo{write down which are the possible classification categories
for Analysis Tools in Python}}

{\inlinetodo{add note in the difficulty of knowing what is the theory
behind some of the most widely used linters}}

The table below summarises the different tools and what they rely on:

\begin{verbatim}
Tool name/source |  Usage  | Analysis base  | Purpose                     |
---------------------------------------------------------------------------
@cannon.._2005   | Embeded | Type analysis  | Type checking               |
Pyflakes[ref]    | Linter  | NA             | Various checks              |
Pylint           | Linter  | NA             | Various checks              |
Pychecker        | Linter  | NA             | Various checks              |
MyPy             | Linter  | Type analysis  | Type checking               |
Enforce*+(defun) |         | Type analysis  | Type checking               |
Sagitta* (defun) |         | Type analysis  | Type checking               |
StyPy            | Linter  | Novel          | Type checking               |
Lyra             |         | AbstrInter     | Various analyses (including Value Analysis)   |
Pytropos         | Linter  | AbstrInter     | Value Analysis (specialised on tensor shapes) |
PyType           | Linter  |                | Type checking and inference |
ICBD (defun)     | Linter  |                | Type checking and inference |
Pyre             | Linter  | AbstrInter     | Type checking               |
Nagini           | Linter  | SMT (Viper)    | Verifier                    |
Typpete          | Linter  | SMT            | Type inference              |
fromherz...2018  | ?       | AbstrInter     | Value Analysis | Verifier   |
monat..2018      | ?       | AbstrInter     | Type Analysis               |
PyAnnotate*+     | Library | Type analysis  | Type inference              |

*: special cases, they are not an static analysis, but dynamic analysis!
?: tool has not been made public
+: uses gradual types

Pyflakes [@pyflakes2005]
Pylint [@thenaultpylint]
PyChecker [@norwitzpychecker]
Pytype https://github.com/google/pytype
ICBD https://github.com/kmod/icbd
Pyrecheck https://github.com/facebook/pyre-check
Sagitta https://github.com/peterhil/sagitta
Nagini https://github.com/marcoeilers/nagini (has paper)
Typpete https://github.com/caterinaurban/Typpete (two references, a thesis and a paper)
monat..2018 paper: Static Analysis by Abstract Interpretation Collecting Types of Python Programs
PyAnnotate https://github.com/dropbox/pyannotate
\end{verbatim}

{\inlinetodo{write a brief description of each one of the tools
presented in the table}}

{\inlinetodo{explain special case of the library \texttt{astroid} (part
of pylint), which is able to perform some Static Value Analysis. It
stops when it cannot determine the value of anything else. Pretty nice.
\texttt{astroid.extract\_node(\textquotesingle{}a=1\textbackslash{}nb=2\textbackslash{}nc=a+b\textbackslash{}nc\textquotesingle{});\ next(\_.infer())}.
https://astroid.readthedocs.io/en/latest/inference.html}}

{\inlinetodo{explain differences between lyra, fromherz..2018 and
Pytropos}}

{\inlinetodo{lyra was not the first to try abstract interpretation for
Python, mention pydrogen https://github.com/lapets/pydrogen}}

\section{Tensor Shape Analysis}\label{tensor-shape-analysis}

{\inlinetodo{add idea from http://nlp.seas.harvard.edu/NamedTensor
Alexander, who suggests building the libraries in a different way so we
do not have some of the problems of working with nameless shapes}}

{\inlinetodo{extend solutions described below with the new libraries you
found, comment below}}

\subsection{Solutions in other languages}%
\label{solutions-in-other-languages}

We may think, If a type system is not expressive enough to capture the
shape of tensors, then we should just start writing code in a language
which does. We may write code in a language like Haskell or even C++,
which have very strong and expressive type systems (it is possible in
C++ to enforce the shape of the tensors with the use of templates and
constraints (C++20)).

In fact, solutions in other languages exists. For example,
\textcite{chen_typesafe_2017} type checks the shape of tensors
(operations) by restricting what can be constructed (via constraints in
the types of objects and functions). Chen's solution uses the powerful
type system of Scala (which runs over the JVM).
\textcite{eaton_statically_2006} does the same, although in an old
version of Haskell. Eaton's encoding of tensor shapes is awkward because
Haskell did not have, at the time, support for natural numbers at the
type level. Haskell also lacked on syntactic sugar for type functions.
\textcite{abe_simple_2015} implement type checking for matrices (aka,
tensors) in OCaml. The library detects the shape of the matrices at
compile time. \textcite{rakic_statically_2012} type check tensor shapes
(they call them matrices) with templates in C++. Templates are only
accessible at compile time, thus the Rakić et al.~library type checks at
compile.

One recent effort to type check tensor shapes in Haskell is the library
\texttt{tensorflow-haskell-deptyped} written by
\textcite{elkin_haskell_2018}. The library is written as a wrapper
around the library port of TensorFlow for Haskell.
\texttt{tensorflow-haskell-deptyped} enforces at the type level how and
which results a operation between compatible tensors is to be performed.

A different path to take is to extend an existing type checking system,
like MyPy, and extend it with dependent types. Dependent types allow us
to carry information from the term level to the type level, i.e.~we can
encode information of our data (available only at runtime) into the type
system. Over this extended type system, we could implement some
restrictions to the operations applied to different types (this is in
fact the strategy taken in the library tensorflow-haskell-deptyped).

\subsection{Theoretical Solutions}\label{theoretical-solutions}

The problem of mismatched shapes is not new, in fact, it is so common
that it has appeared several times with similar solutions
\autocites{arnold_specifying_2010}{griffioen_type_2015}{rink_modeling_2018}{slepak_array-oriented_2014}{trojahner_dependently_2009}.
All solutions, though, are theoretical, they propose type systems which,
if were to be implemented, could warranty type safety, i.e.~no
mismatching of tensor shapes could ever happen at runtime.

The following is a small discussion about the different solutions
proposed to type check tensors found in the literature\footnote{As a
  side note, it is interesting to notice how difficult is to communicate
  ideas in science. All the papers presented in this section hope to
  solve the same (or similar) problem, but they do not reference each
  other, which means that none of them knew what the others were doing.
  The principal reason for this, I think, it is because they all called
  the problem differently.}:

\begin{itemize}
\item
  \textcite{trojahner_dependently_2009}: A paper on type checking of
  arrays. They define all type restrictions using dependent types
  (something that no other paper does). The special keyword of this work
  is \enquote{array programming}.
\item
  \textcite{arnold_specifying_2010}: A paper focused on type checking of
  sparse \enquote{matrix} operations. They define a functional language
  that can be translated into a lower level language, or machine code.
  They give a complete formalization of the algorithms and proofs of
  correctness in Isabelle. The special keyword of this work is
  \enquote{sparse matrix}.
\item
  \textcite{griffioen_type_2015}: A paper focused in type checking and
  inference of arrays in array programming and vector spaces. They
  define a special type system in which tensors are first class
  citizens. The algorithm used for type checking is
  \enquote{Unification} which allows them to infer the type shape of
  arrays. The special keyword of this work is \enquote{array
  programming}.
\item
  \textcite{slepak_array-oriented_2014}: A paper that tries to formalise
  array-oriented programming languages and extend them with unit-aware
  operations. Array-oriented programming languages are languages which
  base all their computations in arrays (like matlab), but they usually
  aren't formalised. The types of arrays not only carry information
  about their shape but also of the unit they carry. The special
  keywords of this work are \enquote{array-oriented programming} and
  \enquote{unit-aware computation}.
\item
  \textcite{rink_modeling_2018}: In this paper, Rink formalises a type
  system to type check the shape of tensors and defines a language to
  use with the type system. It is a custom type system which does not
  require of dependent types. The special keyword of this work is
  \enquote{tensor manipulation language}.
\end{itemize}
