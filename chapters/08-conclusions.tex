\chapter{Conclusions}\label{conclusions}

{\ichange{This is just a DRAFT! REVISE!!}}

In this work I have presented the application of a very powerfull and
very well studied technique for Static Analysis (Abstract
Interpretation), applied to a domain where it has not been applied much
(just a couple of recent papers and applications are using it). In a
brief note the work done includes:

\begin{itemize}
\tightlist
\item
  A formalisation of a subset of Python 3.6
\item
  An Abstract Domain for Python Values
\item
  An Abstract Domain for Python program states
\item
  The semantics of an Abstract Interpreter for Python
\item
  The implementation of the Abstract Interpreter
\item
  A prove of concept domain where, tensor shapes, where the Abstract
  Interpreter could show its power
\item
  A series of tests showcasing the scenarios where the AI is able to
  work fine and show results
\item
  A way for developers to annotate code to improve the accuracy of the
  AI
\end{itemize}

The formalisation of the subset of Python showed one more attempt to
formalise the Python semantics. We focused on how this formalisation
could be used for a very specific library, the numpy library a corner
stone of modern Python.

To implement the Abstract Interpreter, we required the formalisation of
the Abstract Domain which carries the values stored in memory. The State
Abstract Domain is wholy defined by the join operation. The join
operation is the most complex piece in the whole work, it tries to
combine the information of two different states into one that contains
information from both. The values where they differ are then collapsed
properly to a \texttt{Top} value. \texttt{Top} values can be think of as
\texttt{?} types in Gradual Type systems, they are valid types/values
but carry only the information of their existance not what could be they
carrying.

Fortunately, the semantics of the Abstract Interpreter are not very
complicated, they just borrow their capabilities from the State Abstract
Domain and the semantics of Python. It is easy to see how the
formalisation could be extended to work with other values by just
augmenting the Abstract Domain.

\enquote{Extensive} testing showed that the implementation of the
Abstract Interpreter, Pytropos, is able to actually check many common
shape cases. The biggest problem of the implementation relies on the
lack of capabilities it can offer (harshly limited to forgo user-defined
functions and classes).

The Abstract Interpreter also lets itself to extend with type
annotations. Type Annotations are meant to be given by the user, who
always knows what is doing, and can tell the poor AI what the value of
some variable is. This is easily done by replacing the old value by the
new one with the annotation only if the annotation contains for
information about the variable than the value contained on it (ie, hint
\textless{} val).

\section{Future Work}\label{future-work}

Python is a huge and rich language. The amount of Python characteristics
exceeds that of what a single human could try to implement for a work
like this. In fact, implementing an Abstract Interprer can be regarded
to be as hard, or harder, than a regular interpreter.

Much work left to do to try to match the power of Pytropos to make it
usable for the regular developer. I propose the following roadmap to
continue building the Abstract Interpreter:

\begin{itemize}
\tightlist
\item
  Extend Pytropos to include Exception handling (probably, a similar
  approach to that of {\todo{add ref to ai last year python}} could be
  applied here).
\item
  Improve how spliting stores and \textbf{join}ing them latter is
  done. The join operation between stores is very, very costly. It
  requires walking through both graphs/stores at the same time. That is
  extremely expensive when only a variable has changed its value between
  two different stores. This associated cost could be reduced if not a
  whole store is created everytime a new store is need to be created.
  (One way to do this is by using immutable structures for all values in
  the implementation. The current implementation uses the Heap provided
  by CPython underneat but if one where to define ones heap from scratch
  this would not be a problem).
\item
  Extend, again using the ideas put on the paper on ai {\todo{add ref,
  same as above}}, Pytropos to handle break statements.
\item
  Extend the definition of Store from a simple Global scope + heap to
  something that handles different levels of scope. This would allow an
  easier implementation of user defined functions. The function scope of
  Python is middly complex (this is due to the \pycode|compile|-ation
  process to which the parsed code is put to. Python does compile, but
  it does it very, very fast) with four different variable scopes:
  global, local, nonlocal, and class.
\item
  Once the interpreter handles user-defined functions properly,
  extending the interpreter to work with user-defined objects is not a
  big problem. The biggest difficulty is the inherit MRO behind the
  complex multi-classing system of Python.
\end{itemize}

{\inlinetodo{add idea that Pytropos could be run inside an IDE just as
paper: Sulir\_Poruban (2018) - Augmenting Source Code Lines with Sample
Variable Values}} {\inlinetodo{add idea that the parser could be
extended to add holes, or insert Top values, just as done in paper: Omar
et al (2018) - Live Functional Programming with Typed Holes (I wrote
something about it in the SoRI)}}

Besides what is left to do to make Pytropos more powerful, there are
some tasks related to the formality of the work. This work just
presented a formalisation of a subset of Python, an Abstract Domain for
Python, and \enquote{inferred} Abstract Semantics, but it nothing is
ever proved. I built from theories to make something workable, but it
may be good if we try to formalise a little bit more Python. We want
some warranties after all! The things that I consider should be done
next are:

\begin{itemize}
\tightlist
\item
  Give a complete and through formal definition for a subset of Python
  in a proof-assistant language such as Agda, Idris, Coq, Isabelle/HOL,
  or one of the many others.
\item
  Define in the same proof-assistance language the Abstract Domain, the
  properties it must follow, and the inferred semantics it produces.
\item
  Prove that the inferred semantics are in fact \textbf{inferred} from
  the Galois connection of the Abstract Domain.
\end{itemize}

Such a formalisation of Python would likely/hopefully help future
endeavours on proofs and verification.

Notice that any formalisation of Python must take some stance on how
closely it wants to follow the implementation of CPython. CPython has
many, many little undocumented semantic subtleties that trying to write
a full formal definition would probably defeat its purpose, namely, help
other projects where a clean and clear formalisation is necessary.

{\inlinetodo{expand on what you wrote in SoRI}}
