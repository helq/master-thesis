\chapter{Validation and Discussion}\label{validation-and-discussion}

{\ichange{This is just a DRAFT! REVISE!!}}

So, this chapter is about which kind of tests where performed and which
were considered but didn't cut it. The Python Regression tests are very
broad in scope, they use too many builtin capabilities which makes it
hard for a prototype like Pytropos to make any of them work (and that is
maybe why they are the regression tests for Python, after all they need
to check as many characteristics as possible!).

\section{Validation}\label{validation}

{\inlinetodo{Over which set of problems you tested your code. The
problems are pieces of code written by me on my own and other collected
from programs from the internet}}

I created two types of tests: unit regression tests and property-based
tests. The unit tests let us check for any kind of anomalous behaviour
while property-based tests ensure that the abstraction follows the
Python implementation that underlies.

Comparing against the official regression tests is left for future work,
as the number of builtin characteristics does not yet reach a
satisfactory level to check.

\subsection{Unit Tests}\label{unit-tests}

There are a total of 79 unit tests checking various parts of the
Abstract Interpreter. The tests can be categorised into:

\begin{itemize}
\tightlist
\item
  binary operations
\item
  type annotations
\item
  numpy library
\item
  if branching
\item
  while looping
\item
  lists and tuples
\item
  join stores
\end{itemize}

In the table below the result of executing all tests can be seen. The
coverage is not 100\% as Pytropos is still in development. The tests
check not only for what Pytropos is able to do now, but what it should
be able given the description of it given in the document.

\begin{verbatim}
Category     Total lines    Tests     Time analysing   covered  non-covered   Total coverage
Binary ops   120            10        500ms            8        2             80%
\end{verbatim}

\subsection{Property-based Tests}\label{property-based-tests}

Property-based tests that try to disprove a property. For example, a
property of lists is that anytime you add a new element to a list of
size \texttt{n} you will get a list with size \texttt{n+1}. I use
property-based tests to check for correcness of Pytropos against Python
on builtin operations for builtin values.

There are a total of 20 of such tests. Because each one of them is a
property-based tests, they are tested against a 100 values that try to
disprove the property. This kind of testing proved to be extremely
helpful at the start of the development, as the \enquote{property-based
tester} was able to find numerous counter examples to the properties.
Python throws an exception on many cases of bad values, the
\enquote{tester} was able to reliably (at least at the start) to come up
with combinations of values that would go wrong when operated. Some of
the most honorable mentions are:

\begin{itemize}
\tightlist
\item
  Zero-division error on expressions like \texttt{5\ \%\ 0} or
  \texttt{5\ \%\ False}
\item
  Value error on expressions like \texttt{5\ \textless{}\textless{}\ -2}
\item
  Excesive High-memory consumption on expressions like
  \texttt{5\ \textless{}\textless{}\ 10**10}
\item
  Overflow error on expressions like \texttt{10**209\ +\ 2.0}
\end{itemize}

\section{Discussion}\label{discussion}

\inlinetodo{add many, many examples of what the program is able to catch and  what not,
based on the errors above}

\inlinetodo{explain and add a picture of what Pytropos shows when executed}

%--

Pytropos is still in a very young age. It is not able to check many,
many characteristics present in Python code (some of the biggest
culprits are \texttt{for} statement, custom classes and objects,
exception handling and the lack of builtin definitions), but when code
is given to it that it can actually check.

Pytropos is able to check for the many common cases and mistakes that
can be made when working with tensors. It is able to calculate the shape
of tensors in a variety of circumstances (NumPy's \texttt{array} can
evaluate almost anything as an array, therefore detecting the shape of
any value passed to \texttt{array} is an accomplishment in itself. An
example of why this is important can be seen in the library{\todo{add
ref to numpy mypy lib}}, where they define a stub for mypy to check of
the library numpy, and what they do in the case of \texttt{array} is
basically give up).
