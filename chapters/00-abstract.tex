\begin{abstractpage}
\begin{abstract}[Resumen][spanish]

Los tensores, una extensión de los arrays, se usan de manera extensiva en una gran
variedad de problemas en programación. Los tensores son los bloques básicos de
construcción de múltiples frameworks de Aprendizaje de Máquina y son fundamentales en la
definición de modelos de Aprendizaje Profundo. Los linters son herramientas indispensables
para los programadores de hoy en día, ya que estos ayudan a los desarrolladores a revisar
el código antes de ejecutarlo. Aunque los tensores son muy populares no existen tensores
que revisen código con operaciones tensoriales. Dada la gran cantidad de trabajo hecho en
Python con (y sin) tensores, es sorprendente el poco trabajo que se ha hecho en esta área.
La Interpretación Abstracta es una metodología/framework diseñada para analizar código de
forma estática. La idea de Interpretación Abstracta es sobreaproximar de manera "sound" el
resultado de ejecutar una pieza de código sobre todos las posibles entradas del programa.
Una sobreaproximación "sound" asegura que el Interprete Abstracto nunca omitirá un
verdadero negativo, es decir, si una pieza de código no es señalada como incorrecta por el
Interprete Abstracto entonces se puede asumir con seguridad que el código nunca fallará.
El Interprete Abstracto puede ser modificado para que sólo informe acerca de verdaderos
positivos, aunque se pierda la propiedad de "soundness", es decir, el interprete sólo
informa acerca de las partes de código que fallarán sin importar que suceda.

En este trabajo, formalizamos un subconjunto de Python con énfasis en operaciones con
tensores. Nuestra formalización de la semántica de Python está basada en la Referencia
oficial del Lenguage Python. Definimos un Interprete Abstracto y presentamos su
implementación. Mostramos como cada parte del Interprete Abstracto fué definido: su
Dominio Abstracto y semántica abstracta.

Presentamos la estructura de Pytropos, la implementación del Interprete Abstracto.
Pytropos es capaz de revisar las operaciones de arreglos de NumPy teniendo en cuenta
broadcasting y algunas funciones complejas de NumPy como \texttt{array} y \texttt{dot}.
Construimos 74 casos de prueba unitaros, los cuales chequean la capacidad de Pytropos,
además de 20 casos de prueba de propiedades, los cuales chequean que las reglas semánticas
de Pytropos correspondan con la forma en la que Python es ejecutado. Mostramos las
capacidades del Interprete Abstracto por medio de ejemplos de los test unitarios.
\end{abstract}

\cleardoublepage

\begin{abstract}[Abstract][english]
Tensors, an extension of arrays, are widely used in a variety of
programming tasks. Tensors are the building blocks of many modern machine
learning frameworks and are fundamental in the definition of deep
learning models. Linters are indispensable tools for today's developers,
as they help the developers to check code before executing it. Despite
the popularity of tensors, linters for Python that check and flag code
with tensors are nonexistent. Given the tremendous amount of work done
in Python with (and without) tensors, it is quite baffling that little
work has been done in this regard. Abstract Interpretation is a
methodology/framework for statically analysing code. The idea of
Abstract Interpretation is to soundly overapproximate the result of
running a piece of code over all possible inputs of the program. A sound
overapproximation ensures that the Abstract Interpreter will never omit
a true negative, i.e.~if a piece of code is not flagged by the Abstract
Interpreter, then it can be safely assumed that the code will not fail.
The Abstract Interpreter can be modified so that it only outputs true
positives, although losing soundness, i.e.~the interpreter can flag
which parts of the code are going to fail no matter how the code is run.

In this work, we specify a subset of Python with emphasis on tensor
operations. Our operational Python semantics is based on The Python Language Reference. We
define an Abstract Interpreter and present its implementation. We show how each part of the
Abstract Interpreter was built: the Abstract Domains defined and the abstract
semantics.

We present the structure of Pytropos, the Abstract
Interpreter implemented. Pytropos is able to check NumPy array operations
taking into account broadcasting and complex NumPy functions as
\texttt{array} and \texttt{dot}. We constructed 74 unit test cases
checking the capabilities of Pytropos and 20 property test cases
checking compliance with the official Python implementation. We show
what and how many bugs the Abstract Interpreter was able to find.
\end{abstract}
\end{abstractpage}
