\chapter{Background}\label{background}

In this chapter, we explore a couple of motivational topics and ideas
necessary on the development of the Static Value Analysis proposed in
Chapter \ref{appendix-ai-theory}.

\section{Dynamic and Static Analysis}%
\label{dynamic-and-static-analysis}

Static Analysis and Dynamic Analysis are two
disciplines{\todo{really??}} whose ideal is to ensure some property
holds on a piece of code. Static Analysis checks the code without the
need to run the code, while Dynamic Analysis checks the code as it is
executed.

Static Analysis is often associated with compiled Programming Languages.
Compiled Programming Languages require to know how to precisely
translate the operations of the language into machine instructions.
Compiled Programming Languages make use of a variety of Static Analyses
to find out the suitable machine instructions for a piece of code.

Interpreted Programming Languages, on the other hand, find out which
machine instructions to use for a given operation as they encounter the
operations, thus they make heavy use of Dynamic Analyses. Notice that
the distinction between Compiled Programming Languages and Interpreted
Programming Languages does not stem from their use of Static and Dynamic
Analyses but how they process the input files. Compiled Programming
Languages read the totality of the input files and translate the
operations into machine instructions, while Interpreted Programming
Languages do not need to read the whole file before starting to execute
the instructions in them.

Static and Dynamic Typing are two categories of the same kind of
Analysis, Type Analysis \autocite{pierce_types_2002}. Type Analysis
tells compilers and interpreters the type of a variable. Both Static and
Dynamic Typing based Programming Languages have strong and weak points:
Compiled Languages hold stronger warranties for the resulting binary as
they restrict the set of valid programs to those that \enquote{type
check}, and Interpreted Languages allow the programmer to forgo the
usually expensive step of compiling code.

Dynamically Typed, Interpreted Programming Languages (Dynamically Typed
Languages) are excellent tools for prototyping as they let the developer
test their code without waiting for it to compile. However, writing big
pieces of software in a Dynamically Typed Language is often a
challenging task as the type warranties are not as strong as in compiled
programming languages.

Developers are more prone to make a mistake that will only appear once
the code is executed in a Dynamically Typed Languages opposed to
Statically Typed (Compiled) Languages. Dynamic Typing is not ideal for
code that needs to warranty a high degree of reliability. Reliability
defined as the certainty that the code will never fail due to type
mismatches.

Traditional Static Typing can be applied to Dynamically Typed Languages
to acquire the same level of confidence of Statically Typed Languages.
Unfortunately, the restrictions Static Typing induces in the code often
make them to much of a hassle for developers of Dynamic Typed Languages.

Several proposals to make Dynamic Type Systems more robust have appeared
in recent years. Some notable proposals and implementations include:

\begin{itemize}
\tightlist
\item
  Gradual types for python, Mypy \autocite{lehtosalo2016mypy},
\item
  Gradual types for Javascript, Typescript
  \autocites{bierman2014understanding}{hejlsberg2012introducing} and
  Flow \autocite{chaudhuri2016flow},
\item
  Gradual types for ruby, using the library rubydust
  \autocite{an_dynamic_2011}, and
\item
  Refinement types for Ruby \autocite{kazerounian_refinement_2017}.
\end{itemize}

Gradual Typing \autocite{siek_gradual_2006} was proposed as a way to
bridge the gap between Static and Dynamic Typing. The main idea of
Gradual Typing is to enforce Static Typing only to those parts of the
code that the developer cares about.

Gradual Typing adds a new type value to the Type System, \texttt{?}
(named \emph{Any}). In Gradual Typing all variables are of type
\texttt{?} unless the user annotates them with a more precise type like
\texttt{int} or \texttt{float}. Type annotations are optional in Gradual
Typing, as opposed to Static Typing where the type of all variables must
be known (either by annotations or inference). Strictly speaking,
Gradual Typing is a Static Typing Algorithm that simulates Dynamic
Typing by extending the typing rules with the type \texttt{?}. Operating
with \texttt{?} means that we have no idea of the type of the variable
we are working with, exactly what happens on Dynamic Typing.

Refinement Types \autocite{rushby_subtypes_1998} are mainly used for
verification. The idea of Refinement Types is to make sure a piece of
code follows a formal specification. Formal specifications are written
in logic and are translated, usually, into propositional formulae to be
checked by SAT or SMT Solvers.

\section{Python Type Annotations}\label{python-type-annotations}

Since Python 3.5 \autocite{pep484} functions inputs and output can be
annotated with types. Since Python 3.6 \autocite{pep526} variables can
be type annotated too. Type annotations do no modify the type of the
variable they are attached to. Their purpose is only to provide a way to
annotate the type the user believes the variable has. Type annotations
are used by external libraries as MyPy \autocite{lehtosalo2016mypy} and
Enforce \footnote{GitHub project accessible on
  \url{https://github.com/RussBaz/enforce}} to check for correctness of
the code. The Python \texttt{typing} library offers a builtin arange of
variables to type annotate variables. For an in-depth explanation on the
goal of Type Annotations and how are they used in Python refer to
\autocite{pep483}.

\section{Machine Learning and Python}%
\label{machine-learning-and-python}

Machine Learning has been on the rise for the last couple of years since
Deep Learning (DL) exploded in popularity. DL has been able to
outperform the state of the art in many areas in computer science. DL
research consists of finding new models that perform a certain task as
well as possible (sometimes even better than humans).

A trained DL model represents the work of many hundreds of hours (often
thousand or more) of computing and coding time. Hundreds of DL models
have been released to the public by research institutes, the academia,
and the industry alike. Given the high amount of competition, it is
fundamental for DL practitioners to be able to test many DL models as
fast as possible. Trying to find a better model before than others is a
fierce fight that drains many hours of laborious thinking, coding and
debugging.

Nowadays, most papers on Machine Learning (ML) and Deep Learning (DL)
use Python to define their models. In fact, most DL libraries have an
interface to Python. Some of the many Python libraries at disposition
for programmers are: TensorFlow \autocite{abadi_tensorflow_2016},
PyTorch \autocite{paszke2017pytorch}, (the now defunct) Theano
\autocite{bergstra2011theano}, and pycaffe \autocite[as part of the
Caffe framework]{jia2014caffe}.

Python has become just the right tool for prototyping models, and
therefore, Python (and other Dynamic Typed Languages like Lua) has been
the primordial playground for DL enthusiasts in the last couple of
years.

\section{Tensors}\label{tensors}

Array computation is a convenient abstraction to write code that
requires the manipulation blocks of variables. The NumPy library
presents an example of what can be done with it. Notice how the two
pieces of code below perform the same operation (\enquote{normalization}
of the values in the matrix) but NumPy does it in just fewer lines:

\begin{Shaded}
\begin{Highlighting}[]
\ImportTok{import}\NormalTok{ numpy }\ImportTok{as}\NormalTok{ np}
\NormalTok{A }\OperatorTok{=}\NormalTok{ np.array([}
\NormalTok{  [}\FloatTok{1.}\NormalTok{, }\FloatTok{2.}\NormalTok{, }\FloatTok{4.}\NormalTok{],}
\NormalTok{  [}\FloatTok{6.}\NormalTok{, }\FloatTok{3.}\NormalTok{, }\FloatTok{2.}\NormalTok{],}
\NormalTok{  [}\FloatTok{6.}\NormalTok{, }\FloatTok{9.}\NormalTok{, }\FloatTok{5.}\NormalTok{],}
\NormalTok{])}

\NormalTok{A }\OperatorTok{/=}\NormalTok{ np.}\BuiltInTok{max}\NormalTok{(A)}
\end{Highlighting}
\end{Shaded}

\begin{Shaded}
\begin{Highlighting}[]
\NormalTok{A }\OperatorTok{=}\NormalTok{ [}
\NormalTok{  [}\FloatTok{1.}\NormalTok{, }\FloatTok{2.}\NormalTok{, }\FloatTok{4.}\NormalTok{],}
\NormalTok{  [}\FloatTok{6.}\NormalTok{, }\FloatTok{3.}\NormalTok{, }\FloatTok{2.}\NormalTok{],}
\NormalTok{  [}\FloatTok{6.}\NormalTok{, }\FloatTok{9.}\NormalTok{, }\FloatTok{5.}\NormalTok{],}
\NormalTok{]}

\NormalTok{m }\OperatorTok{=} \BuiltInTok{max}\NormalTok{(}\BuiltInTok{map}\NormalTok{(}\BuiltInTok{max}\NormalTok{, A))}
\ControlFlowTok{for}\NormalTok{ i }\KeywordTok{in} \BuiltInTok{range}\NormalTok{(}\DecValTok{3}\NormalTok{):}
  \ControlFlowTok{for}\NormalTok{ j }\KeywordTok{in} \BuiltInTok{range}\NormalTok{(}\DecValTok{3}\NormalTok{):}
\NormalTok{    A[i][j] }\OperatorTok{/=}\NormalTok{ m}
\end{Highlighting}
\end{Shaded}

Multidimensional arrays, arrays which are indexed by not one but two
integers, are often called tensors. We take in this work the same
approach of calling all--arrays, matrices and tensors--the same way,
tensors. If a tensor has one dimension it is equivalent to an array, and
if it has two dimensions it is equivalent to a matrix.

Operating with tensors is often syntactically simpler than using looping
structures to operate with blocks of memory, i.e.~writing \texttt{a+b},
where \texttt{a} and \texttt{b} are tensors is simpler than
\texttt{{[}a{[}i{]}+b{[}i{]}\ for\ i\ in\ range(len(a)){]}}. Besides,
\texttt{a+b} even works for tensors with different (but compatible)
shapes.

\section{NumPy: Library for tensor computation}
\label{numpy-library-for-tensor-computation}

The primary purpose of this work is to build a Static Value Analysis to
check for tensors and tensors operations. NumPy
\autocite{oliphant2006guide} is a very widely used library to operate
with tensors and arrays, in fact, many other libraries are built on top
of NumPy.

NumPy tensors are called arrays. An array can be created out of almost
anything, almost any type of value can be interpreted into an array by
NumPy\footnote{Notice that this represents a challenge for Static Type
  Analysis as it requeries the type of a variable to depend on the
  contents of the variable. Fortunately calculating the contents of a
  variable is what Abstract Interpretation does. An Abstract Interpreter
  allows us to implement/replicate the whole semantics of a library in a
  similar way at how it is defined in the original library}. For
example, all of the following variables hold a valid NumPy array:

\begin{Shaded}
\begin{Highlighting}[]
\NormalTok{a }\OperatorTok{=}\NormalTok{ np.array(}\StringTok{"some text"}\NormalTok{)       }\CommentTok{# array of chars (8-bit numbers)}
\NormalTok{b }\OperatorTok{=}\NormalTok{ np.array([[}\DecValTok{1}\NormalTok{, }\DecValTok{2}\NormalTok{], [}\DecValTok{2}\NormalTok{, }\DecValTok{4}\NormalTok{]])  }\CommentTok{# array of ints of shape (2, 2)}
\NormalTok{c }\OperatorTok{=}\NormalTok{ np.array([[}\DecValTok{1}\NormalTok{, }\DecValTok{2}\NormalTok{], [}\DecValTok{2}\NormalTok{]])     }\CommentTok{# array of objects of shape (2,)}
\NormalTok{d }\OperatorTok{=}\NormalTok{ np.array([(}\DecValTok{1}\NormalTok{, }\DecValTok{2}\NormalTok{), [}\DecValTok{2}\NormalTok{, }\DecValTok{5}\NormalTok{]])  }\CommentTok{# array of ints of shape (2, 2)}
\NormalTok{e }\OperatorTok{=}\NormalTok{ np.array([[}\DecValTok{1}\NormalTok{, }\DecValTok{2}\NormalTok{], \{}\DecValTok{2}\NormalTok{, }\DecValTok{5}\NormalTok{\}])     }\CommentTok{# array of objects of shape (2,)}
\end{Highlighting}
\end{Shaded}

The shape of a NumPy array can be changed without changing its contents:

\begin{Shaded}
\begin{Highlighting}[]
\NormalTok{a }\OperatorTok{=}\NormalTok{ np.arange(}\DecValTok{50}\NormalTok{)}
\NormalTok{a }\OperatorTok{=}\NormalTok{ a.reshape((}\DecValTok{1}\NormalTok{,}\OperatorTok{-}\DecValTok{1}\NormalTok{,}\DecValTok{2}\NormalTok{,}\DecValTok{5}\NormalTok{))}
\end{Highlighting}
\end{Shaded}

Reshaping preserves the number of elements in an array. If an
\texttt{-1} is found, it will be taken as a wild card and the missing
dimension will be calculated to keep the original shape.

Numpy has this little, nice trick to handle operations that would
require reshaping or copying an array when operating with two arrays
with different, but similar, shapes. The trick is called broadcasting,
and can be better understood with an example:

\begin{Shaded}
\begin{Highlighting}[]
\NormalTok{a }\OperatorTok{=}\NormalTok{ np.zeros((}\DecValTok{10}\NormalTok{, }\DecValTok{3}\NormalTok{, }\DecValTok{4}\NormalTok{)) }\OperatorTok{+} \DecValTok{2}
\NormalTok{b }\OperatorTok{=}\NormalTok{ np.arange(}\DecValTok{3}\NormalTok{).reshape((}\DecValTok{3}\NormalTok{, }\DecValTok{1}\NormalTok{))}
\NormalTok{c }\OperatorTok{=}\NormalTok{ a }\OperatorTok{*}\NormalTok{ b}
\end{Highlighting}
\end{Shaded}

Notice that the code is valid and evaluates in Python. The shape of
\texttt{a} is \texttt{(10,\ 3,\ 4)} because adding an array of any shape
to an \texttt{int} gives us back an array with the same shape, i.e.~we
have added a scalar to every element of a tensor. \texttt{b} has a shape
of \texttt{(3,\ 1)} and the shape of \texttt{c} is
\texttt{(10,\ 3,\ 4)}. Broadcasting \enquote{generalises} the rule of
operating with a scalar (as it is done in the first line). The rule
works by extending with \texttt{1}'s the left side of the smaller shape
until both are equal, and then checking if the dimensions of both shapes
are compatible. Two dimensions are compatible if are the same or one of
them is \texttt{1}. For a deeper explanation on broadcasting take a look
at \url{https://docs.scipy.org/doc/numpy/user/basics.broadcasting.html}.

\section{Abstract Interpretation}\label{abstract-interpretation}

In this section, a rather informal description of Abstract
Interpretation is given. For an in-depth explanation of Abstract
Interpretation refer to \autocites[Chapters 1 and
4]{nielson2015principles}{cousot_abstract_1977}{nipkow_abstract_2014}.

\subsection{Informal Introduction}\label{informal-introduction}

Static Analysis includes a broad assortment of techniques with the
purpose of verifying some property in the code. It may be trying to
prove some invariant in the code, like there is never a segmentation
fault, the code does what it says it does even when ran in concurrently,
or the codes terminates (given some restrictions
\autocite{urban_static_2015}).

Abstract Interpretation was conceived \autocite{cousot_abstract_1977} as
a way to approximate the result of all possible executions of a
computation. Specifically, Abstract Interpretation computes an
\textbf{over}approximation of some property of the program. To explain
what an overapproximation is consider the following piece of code:

\begin{Shaded}
\begin{Highlighting}[]
\NormalTok{a }\OperatorTok{=} \BuiltInTok{int}\NormalTok{(}\BuiltInTok{input}\NormalTok{())}
\NormalTok{a }\OperatorTok{=} \BuiltInTok{abs}\NormalTok{(a)}
\NormalTok{b }\OperatorTok{=} \DecValTok{2} \OperatorTok{*}\NormalTok{ a}
\end{Highlighting}
\end{Shaded}

Assuming the user inputs a valid number and the code runs as intended,
we know at the end of the computation that: \texttt{a} holds a
non-negative integer, and \texttt{b} holds a non-negative integer
multiple of 2. An overapproximation of this computation is saying that
\texttt{a} and \texttt{b} are both non-negative integers (notice that we
do not say that \texttt{b} is even).

There is an inherent tradeoff between the precision of an
overapproximation, and the amount of storage and processing the
overapproximation requires. The most precise method is to store in a set
all the possible values a computation may take but it is often
impossible to do. The example above asks us to store an infinite set of
numbers, which is impossible.

What is very interesting about Abstract Interpretation is that it is
based on solid theory. If one applies it correctly, a Static Analysis
based on Abstract Interpretation is warrantied to be sound, i.e.~it can
warranty there will never be a miss or false negatives. Although given
the nature of overapproximating, the Static Analysis may produce a load
of false positives. The amount of false positives depends on the
specific overapproximation used.

How is the overapproximation calculated and which tools do we have at
our disposal to compute it? The following subsections present briefly
the important concepts behind Abstract Interpretation. For an in-depth
explanation of Abstract Interpretation refer to \autocites[Chapters 1
and
4]{nielson2015principles}{cousot_abstract_1977}{nipkow_abstract_2014}.

\subsection{Ingredients for an Abstract Interpreter}
\label{ingredients-for-an-abstract-interpreter}

Before we start we need some groundwork. First of all, as always in
Programming Languages, we need a language to analyse together with its
formal semantics. We also need a property we want to analyse.

Specific for this work, the property to analyse will be the value of
variables throughout the execution of the program. i.e.~Value Analysis.
Abstract Interpretation can be used to analyse any other property of
programs, some examples include: memory consumption, functional
properties, thread safety, computation traces, and termination.

Before we can analyse a piece of code, we require to know the language
to analyse. For Value Analysis we require the following description of
the language:

\begin{itemize}
\tightlist
\item
  The syntax of the language (how to write things),
\item
  The values the language handles (for example, integers, booleans and
  floats),
\item
  The state of a program (how to group the values and their names), and
\item
  The concrete semantics of the language (the rules that tell us how to
  evaluate the code)
\end{itemize}

Given these we can define:

\begin{itemize}
\tightlist
\item
  An Abstract Domain for the values in the language (the
  overapproximation used to represent values),
\item
  An Abstract Domain for the state of the program (overapproximates the
  state of the program), and
\item
  The abstract semantics induced from the Abstract Domain and the
  concrete semantics.
\end{itemize}

\subsection{Value Abstract Domain}\label{value-abstract-domain}

Our goal with Abstract Interpretation is to overapproximate the result
of running a program on all possible inputs. Remember that a variable
holds a single value at the time, so a naïve approach to run over all
possible inputs is to extend what a variable can hold.

Suppose a variable can hold a set of values, not a single value, and we
extend the semantics of a language to operate over all combinations of
values any time an operation is made between two variables. If we do
this, we will be able to calculate all possible states a program may
arrive. For example, consider:

\begin{Shaded}
\begin{Highlighting}[]
\NormalTok{cond1 }\OperatorTok{=} \BuiltInTok{bool}\NormalTok{(}\BuiltInTok{input}\NormalTok{())  }\CommentTok{# Stdin input}
\NormalTok{cond2 }\OperatorTok{=} \BuiltInTok{bool}\NormalTok{(}\BuiltInTok{input}\NormalTok{())}

\ControlFlowTok{if}\NormalTok{ cond1:}
\NormalTok{  a }\OperatorTok{=} \DecValTok{0}
\NormalTok{  b }\OperatorTok{=} \DecValTok{20}
\ControlFlowTok{elif}\NormalTok{ cond2:}
\NormalTok{  a }\OperatorTok{=} \DecValTok{2}
\NormalTok{  b }\OperatorTok{=} \DecValTok{6}
\ControlFlowTok{else}\NormalTok{:}
\NormalTok{  a }\OperatorTok{=} \DecValTok{1}
\NormalTok{  b }\OperatorTok{=} \DecValTok{9}
\end{Highlighting}
\end{Shaded}

After running the code, we know that \texttt{a} \(\in \{0, 1, 2\}\) and
\texttt{b} \(\in \{6, 9, 20\}\). Now, if we want to compute \texttt{a+b}
we must compute all possible results of computing with the values from
\texttt{a} and \texttt{b}, which results in \texttt{a+b}
\(\in \{6, 9, 20, 7, 10, 21, 8, 11, 22\}\). Notice how the result of
calculating \texttt{a+b} overapproximates the real values \texttt{a+b}
can take, i.e. \texttt{a} and \texttt{b} are assumed to be independent
of each other so we get way more possible values for \texttt{a+b} than
can actually be computed (we overapproximate).

It is clear that computing with sets of values is expensive, even
impossible at times, e.g.~the set \(\{a \in \mathbb{N} | a \leq 0\}\)
cannot be stored in memory.

What we can do to not compute all possible computations is to use an
abstraction that lets us overapproximate the values contained in the
sets. For example, if we use interval arithmetic to approximate the sets
we get:

\texttt{a} \(\in \left[0, 2\right]\), \texttt{b}
\(\in \left[6, 20\right]\), and \texttt{a+b} \(\in \left[6, 22\right]\)

Which can be easily stored and manipulated (we only require two numbers
to encode an interval). Notice how any abstraction will make us lose
some precision. This is the inherent tradeoff of computing using
Abstract Interpretation, we lose precision.

Notice that overapproximating the value of variables can lead us often
to think that a piece of code may be erroneous even if it does not.
Consider:

\begin{Shaded}
\begin{Highlighting}[]
\NormalTok{b }\OperatorTok{=} \BuiltInTok{bool}\NormalTok{(}\BuiltInTok{input}\NormalTok{())}

\ControlFlowTok{if}\NormalTok{ b:}
\NormalTok{  a }\OperatorTok{=} \DecValTok{2}
\NormalTok{  b }\OperatorTok{=} \DecValTok{-1}
\ControlFlowTok{else}\NormalTok{:}
\NormalTok{  a }\OperatorTok{=} \DecValTok{6}
\NormalTok{  b }\OperatorTok{=} \DecValTok{3}

\NormalTok{c }\OperatorTok{=}\NormalTok{ a }\OperatorTok{/}\NormalTok{ b  }\CommentTok{# c is either -2 or 2}
\end{Highlighting}
\end{Shaded}

It is simple for us to run this example in all possible inputs, as there
are only two, an notice that the code never fails. If \texttt{b} is
\texttt{True}, then we know that \texttt{c} is \texttt{-2}, and if
\texttt{b} is \texttt{False}, then we know that \texttt{c} is
\texttt{2}. We know that no matter the input, the code will never fail!
But when we abstractly interpret the code using intervals as an
overapproximation, we get that \texttt{a} \(\in [2,6]\) and \texttt{b}
\(\in [-1,3]\), so when we try to run \texttt{a\ /\ b} we are alerted
that we may be dividing by zero\footnote{dividing by zero throws an
  Exception in Python}.

When we abstractly interpret a piece of code and get an error, the error
may not exist.

On the other hand, if we abstractly interpret a piece of code and find
no error in it, we can be sure that the code will never fail because we
have tested the code on an \textbf{over}approximation of all possible
values\footnote{assuming no builtin value has been changed prior to the
  execution of the code}. This property is called \emph{soundness} and
it is central to verification and other areas of Static Analysis.

Notice that we do not care about soundness in this work. Our goal is to
build a tool that does not overwhelm the common developer but helps them
to find \emph{true} bugs! i.e.~we do not want false positives just as
many true positives as possible.

Fortunately, if the overapproximation we are using allows us to
distinguish between single values, e.g.~the overapproximation can tell
us that \texttt{a} has been set to the value \texttt{5} after the
assignation \texttt{a\ =\ 5}\footnote{The sign overapproximation (or
  Sign Interval Abstract Domain) has only three posible values:
  \texttt{-}, \texttt{0} and \texttt{+}, while the interval
  overapproximation (or Interval Abstract Domain) contains any interval
  \([a,b]\) with \(a,b \in \mathbb{R}\). Therefore, if we use the
  Interval Abstract Domain we are able to tell when we know a variable
  has a specific value rather than many. On the other hand it is
  imposible for use to determine a unique value using Sign Interval
  Domain other than the case \texttt{0}.}, then it is possible for us to
check if two single values are equal or different. If we are able to
assert with certainty that two values are equal or different, then we
are able to find true positive comparisons (oppossed to maybe
\texttt{True} comparisons)\footnote{Consider the result of
  \texttt{{[}4,4{]}\ ==\ {[}3,3{]}} (\texttt{False}) and the result of
  \texttt{{[}3,8{]}\ ==\ {[}2,3{]}} (it may be \texttt{True} or
  \texttt{False})}.

The formal definition of an overapproximation is an Abstract Domain. An
Abstract Domain is composed of a lattice, a Galois connection between a
set of values and the lattice, and widening and narrowing operators.

A lattice is a partially ordered set \((L, \le)\) with the following
properties:

\begin{itemize}
\tightlist
\item
  Every two elements \(a, b \in L\) have a lower bound (\(u \in L\) such
  that \(u \le a\) and \(u \le b\)). The operation of finding the lower
  bound between two elements is called \textbf{merge} and is denoted by
  \(a \sqcap b\).
\item
  Every two elements \(a, b \in L\) have an upper bound (\(u \in L\)
  such that \(a \le u\) and \(b \le u\)). The operation of finding the
  upper bound between two elements is called \textbf{join} and is
  denoted by \(a \sqcup b\).
\item
  There is an element bigger than any other (\(u \in L\) such that
  \(\forall x \in L . x \le u\)), and it is denoted by \(\top\).
\item
  There is an element smaller than any other (\(u \in L\) such that
  \(\forall x \in L . u \le x\)), and it is denoted by \(\bot\).
\end{itemize}

Given two lattices,
\((L_1, \le{}_{L_1}, \sqcap{}_{L_1}, \sqcup{}_{L_1}, \top{}_{L_1}, \bot{}_{L_1})\)
and
\((L{}_2, \le{}_{L_2}, \sqcap{}_{L_2}, \sqcup{}_{L_2}, \top{}_{L_2}, \bot{}_{L_2})\),
a Galois connection is a pair of functions \((\alpha, \gamma)\) such
that:

\begin{itemize}
\tightlist
\item
  \(\alpha : L_1 \rightarrow L_2\). \(\alpha\) is called the abstraction
  function.
\item
  \(\gamma : L_2 \rightarrow L_1\). \(\gamma\) is called the
  concretisation function.
\item
  \(\forall a \in L_1, b \in L_2 : \alpha(a) \le{}_{L_2} b \iff a \le{}_{L_1} \gamma{b}\).
\end{itemize}

\(L_1\) is often called the Concrete Domain and \(L_2\) is called the
Abstract Domain. Because the Concrete Domain is always the same, the set
of possible values a variable can have with \(\in\) as an order
operator, we concentrate our efforts in the Abstract Domains.

\emph{Notation}: It is customary to use the symbol \(\#\) to refer to
the functions and semantics associated with the Abstract Domain, and the
lack of it as the functions and semantics of the Concrete Domain. For
example, if \(Int\) is the set of all integers then \(Int^{\#}\) is an
Abstract Domain for \(Int\), e.g.~given the Concrete Domain
\((\mathcal{P}(Int), \subseteq, \cap, \cup, Int, \emptyset)\) we define
the Abstract Domain
\((Int^{\#}, \le^{\#}, \sqcap^{\#}, \sqcup^{\#}, \top^{\#}, \bot^{\#}, \alpha, \gamma)\),
where \(\alpha : Int \rightarrow Int^{\#}\) and
\(\gamma : Int^{\#} \rightarrow Int\).

Coming back to the intervals example from before. Given the Concrete
Domain of integers
\((\mathcal{P}(Int), \subseteq, \cap, \cup, Int, \emptyset)\), we can
define the Intervals Abstract Domain \(\text{Int}^{\#}\) as
\((\text{Int}^{\#}, \le^{\#}, \sqcap^{\#}, \sqcup^{\#}, [-\infty,\infty], \emptyset, \alpha, \gamma)\)
where:

\begin{itemize}
\tightlist
\item
  Every element of \(\text{Int}^{\#}\) is an interval \([a, b]\) with
  \(a, b \in Int \cup \{-\infty,\infty\}\).
\item
  \([a, b] \le^{\#} [c, d] \iff c \le a \wedge b \le d\).
\item
  \([a, b] \sqcap^{\#} [c, d] = [max(a,c), min(b, d)]\).
\item
  \([a, b] \sqcup^{\#} [c, d] = [min(a,c), max(b, d)]\).
\item
  \(\alpha(I) = [min(I), max(I)]\) (Remember that \(I\) is a set of
  integers).
\item
  \(\gamma([a, b]) = \{i \in Int : a \le i \le b\}\).
\end{itemize}

The goal of an Abstract Domain is to give us a very powerful tool, the
Galois Connection. The abstraction function allows us to take any value
in our language and transport it to an overapproximation of our
choosing, and the concretisation function allows us to take abstract
values (overapproximations) and operate with them back in the semantics
of our language.

Given, for example, the Intervals Abstract Domain, we can now calculate
the result of computing the following piece of code:

\begin{verbatim}
if _:
  a = 2
else:
  a = 4
a -= 6
\end{verbatim}

We know that the value of \texttt{a} in the \emph{then} branch of the of
the \emph{if} statement is equal to \texttt{{[}2,2{]}} and
\texttt{{[}4,4{]}} in the \emph{else} branch. The value of \texttt{a}
after the execution of the \emph{if} statement will be
\texttt{{[}2,2{]}} \(\cup{}\) \texttt{{[}4,4{]}} \(=\)
\texttt{{[}2,4{]}}. At the end of the execution the value of \texttt{a}
lie inside the interval \texttt{{[}-4,-2{]}}.

Notice that with a (Value) Abstract Domain, we can overapproximate the
value of a single variable but we cannot warranty \textbf{termination}
of the evaluation of the code. Take for example:

\begin{Shaded}
\begin{Highlighting}[]
\NormalTok{a }\OperatorTok{=} \DecValTok{4}
\ControlFlowTok{while} \KeywordTok{not}\NormalTok{ condition(a):}
\NormalTok{  a }\OperatorTok{+=} \DecValTok{3}
\end{Highlighting}
\end{Shaded}

Unrolling the loop, we get:

\begin{Shaded}
\begin{Highlighting}[]
\NormalTok{a }\OperatorTok{=} \DecValTok{4}     \CommentTok{# a \textbackslash{}in [4,4]}

\NormalTok{  a }\OperatorTok{+=} \DecValTok{3}  \CommentTok{# a \textbackslash{}in [4,7]}
\NormalTok{  a }\OperatorTok{+=} \DecValTok{3}  \CommentTok{# a \textbackslash{}in [4,10]}
\NormalTok{  a }\OperatorTok{+=} \DecValTok{3}  \CommentTok{# a \textbackslash{}in [4,13]}
\NormalTok{  a }\OperatorTok{+=} \DecValTok{3}  \CommentTok{# a \textbackslash{}in [4,16]}
\NormalTok{  ...}
  \CommentTok{# a \textbackslash{}in [4, inf]}
\end{Highlighting}
\end{Shaded}

Because we do not know when \texttt{condition(a)} will be met, it is
impossible for us to know how many times will the \texttt{while}'s body
be executed, an so we know that \texttt{a} may be either 4, 7, 11, or
any other integer. An Abstract Interpreter is an interpreter and
therefore will run the body of the while loop forever if we do not do
something to terminate it.

Applying a series of operations (the body of the while loop) over and
over may never terminate. Thus we have the need a mechanism that allows
us to ensure the evaluation of a loop will eventually terminate.
Introduce the \emph{widening} operator.

The idea of a widening operator is to make a series of ascending values
reach a fixed point in a finite amount of time.

Consider the (increasing) sequence from before\footnote{Notice that:
  \texttt{{[}4,4{]}} \(\le\) \texttt{{[}4,7{]}} \(\le\)
  \texttt{{[}4,11{]}}}:

\begin{verbatim}
[4,4] [4,7] [4,10] [4,13] [4,16] [4,19] [4,22] [4,25] ...
\end{verbatim}

A widening operator is an operator, \(\phi{}\), that takes two
intervals, \(a\) and \(b\), and outputs a new interval,
\(c = a \phi b\), such that the new interval contains the old intervals
\(a, b \le c\). The widening operator must warranty that when applied
over and over an increasing sequence gives us a sequence that will,
after a finite amount of steps, find a fixed point (stabilizes). Take
for example the following widening operator:

\[[a, b] \phi [c, d] =
     \begin{cases}
       [min(a,c), \infty]\text{, } & \text{if } b > 15 \\
       [min(a,c), max(b,d)]\text{, } & \text{otherwise}
     \end{cases}\]

Applying the operator over our sequence over an over we will get:

\begin{verbatim}
[4,7] [4,10] [4,13] [4,16] [4,inf] [4,inf] [4,inf] ...
\end{verbatim}

And we found a fixed point, namely, \texttt{{[}4,inf{]}}. We have
arrived now at the top of the latter.

Notice how the cutoff of the sequence must be defined beforehand (in our
case any interval with a value bigger than \texttt{15} defaults to
\(\infty{}\)). This is a little bit annoying as one would like for the
Abstract Interpreter to work under any condition, but this is the price
to pay for termination, we need to set some parameters by hand.

There is another operator called the narrowing operator. Its purpose is
to climb down the latter and get a more precise approximation. Notice
that not all Abstract Domains require the definition of widening and
narrowing operators as not all of them have infinite increasing or
decreasing sequences. And it is the case in this work that none of them
applies, there are no infinite increasing or decreasing sequences in the
Abstract Interpreter defined in this work.

For a deeper discussion on all numerical Abstract Domains available see
\textcite{mine_weakly_2004}.

\subsection{State Abstract Domain}\label{state-abstract-domain}

The Abstract Domains presented before lets us analyse one variable at
the time. To analyse the state of the program we define an Abstract
Domain for the state of the program. How to do this? Well, if we assume
that the language allows no aliasing, then we can build the State
Abstract Domain as:

\begin{verbatim}
State: Var -> Val
State#: Var -> Val#
\end{verbatim}

If the \texttt{State} of the program is defined as a function from
\texttt{Var}s into their values (\texttt{Var\ -\textgreater{}\ Val}),
and \texttt{Val\#} is the Abstract Domain for values in the languages,
then the function \texttt{Var\ -\textgreater{}\ Val\#} is an Abstract
Domain\footnote{For a deeper look and a proof on this statement see
  \textcite{nielson2015principles}, subchapter 4.4.} for the state of
the program.

As an example consider:

\begin{Shaded}
\begin{Highlighting}[]
\DataTypeTok{int}\NormalTok{ main() \{}
  \DataTypeTok{int}\NormalTok{ cond1, cond2, a, b, c;}
  \BuiltInTok{std::}\NormalTok{cin >> cond1 >> cond2;}

  \ControlFlowTok{if}\NormalTok{ (cond1) \{}
\NormalTok{    a = }\DecValTok{0}
\NormalTok{    b = }\DecValTok{20}
\NormalTok{  \} }\ControlFlowTok{else} \ControlFlowTok{if}\NormalTok{ (cond2) \{}
\NormalTok{    a = }\DecValTok{2}
\NormalTok{    b = }\DecValTok{6}
\NormalTok{  \} }\ControlFlowTok{else}\NormalTok{ \{}
\NormalTok{    a = }\DecValTok{1}
\NormalTok{    b = }\DecValTok{9}
\NormalTok{  \}}
\NormalTok{  c = a + b;}
  \ControlFlowTok{return} \DecValTok{0}\NormalTok{;}
\NormalTok{\}}
\end{Highlighting}
\end{Shaded}

Note: This example is written in C++ because we can assume all variables
to have a set type, a unique type. In C++, a variable always has a value
even if none is given to it. In Python, a variable can have no value, or
it can be undefined (as it happens when a variable is deleted).

Suppose that the State Abstract Domain for this piece of code is a
function from the set \(\{cond1, cond2, a, b, c\}\) to the Interval
Abstract Domain. If we evaluate the code line by line we will get:

\begin{Shaded}
\begin{Highlighting}[]
\CommentTok{// T: means Top}
\DataTypeTok{int}\NormalTok{ main() \{}
  \DataTypeTok{int}\NormalTok{ cond1, cond2, a, b, c;}
  \CommentTok{// \{'cond': T, 'cond2': T, 'a': T, 'b': T, 'c': T\}}
  \BuiltInTok{std::}\NormalTok{cin >> cond1 >> cond2;}
  \CommentTok{// \{'cond': T, 'cond2': T, 'a': T, 'b': T, 'c': T\}}

  \ControlFlowTok{if}\NormalTok{ (cond1) \{}
\NormalTok{    a = }\DecValTok{0}
    \CommentTok{// \{'cond': T, 'cond2': T, 'a': 0, 'b': T, 'c': T\}}
\NormalTok{    b = }\DecValTok{20}
    \CommentTok{// \{'cond': T, 'cond2': T, 'a': 0, 'b': 20, 'c': T\}}
\NormalTok{  \} }\ControlFlowTok{else}\NormalTok{ \{}
    \ControlFlowTok{if}\NormalTok{ (cond2) \{}
\NormalTok{      a = }\DecValTok{2}
      \CommentTok{// \{'cond': T, 'cond2': T, 'a': 2, 'b': T, 'c': T\}}
\NormalTok{      b = }\DecValTok{6}
      \CommentTok{// \{'cond': T, 'cond2': T, 'a': 2, 'b': 6, 'c': T\}}
\NormalTok{    \} }\ControlFlowTok{else}\NormalTok{ \{}
\NormalTok{      a = }\DecValTok{1}
      \CommentTok{// \{'cond': T, 'cond2': T, 'a': 1, 'b': T, 'c': T\}}
\NormalTok{      b = }\DecValTok{9}
      \CommentTok{// \{'cond': T, 'cond2': T, 'a': 1, 'b': 9, 'c': T\}}
\NormalTok{    \}}
    \CommentTok{// \{'cond': T, 'cond2': T, 'a': [1,2], 'b': [6,9], 'c': T\}}
\NormalTok{  \}}
  \CommentTok{// \{'cond': T, 'cond2': T, 'a': [0,2], 'b': [6,20], 'c': T\}}
\NormalTok{  c = a + b;}
  \CommentTok{// \{'cond': T, 'cond2': T, 'a': [0,2], 'b': [6,20], 'c': [6,22]\}}
  \ControlFlowTok{return} \DecValTok{0}\NormalTok{;}
\NormalTok{\}}
\end{Highlighting}
\end{Shaded}

Excellent! We have done it! We have the idea of how to build an Abstract
Interpreter for a language with only one type of variable, a global
scope, and no aliasing.

\subsection{Abstract Semantics}\label{abstract-semantics}

Now we have our ingredients all ready: We have a language with its
concrete semantics and State Abstract Domain. What we have left to do is
to define the abstract semantics of the language. The abstract semantics
work on the State Abstract Domain, where as the concrete semantics work
on the state of the program.

For this, we need the Galois connections defined in the State Abstract
Domain, the abstraction \(\alpha\) and concretisation \(\gamma\)
functions. We can infer the abstract semantics with the following little
formula:

\[f^\# = \alpha \cdot f \cdot \gamma\]

where \(f\) is a function (a rule) in the semantics of the language.

The idea is simple, if \(f\) is defined properly defined then we can
compose it with the abstraction and concretisation functions and we will
get a function \(f^\#\) which operates over the State Abstract Domain.

This step is important if we want to prove the Abstract Interpreter to
be sound and work as expected. For a deeper discussion on how to infer
the abstract semantics see \textcite{nielson2015principles}, chapter 4.

\subsection{Missing Bits}\label{missing-bits}

A robust Abstract Interpreter should be able to do more than just what
has been said here.

\textcite{mine_weakly_2004} extends an Abstract Interpreter with the
backward assignment. The backward assignment is meant to restrict the
possible values a variable may have when it enters an \emph{if}
statement. Consider the following piece of C++ (\texttt{a} and
\texttt{b} are integers):

\begin{Shaded}
\begin{Highlighting}[]
\CommentTok{// \{'a': [2,4], 'b': Top\}}
\ControlFlowTok{if}\NormalTok{ (a < }\DecValTok{3}\NormalTok{) \{}
  \CommentTok{// Appling backward assignment. `a<3 and a in [2,4]`}
  \CommentTok{// \{'a': 2, 'b': Top\}}
\NormalTok{  b = a;}
  \CommentTok{// \{'a': 2, 'b': 2\}}
\NormalTok{\} }\ControlFlowTok{else}\NormalTok{ \{}
  \CommentTok{// Appling backward assignment. `a>=3 and a in [2,4]`}
  \CommentTok{// \{'a': [3,4], 'b': Top\}}
\NormalTok{  b = a - }\DecValTok{2}\NormalTok{;}
  \CommentTok{// \{'a': [3,4], 'b': [1,2]\}}
\NormalTok{\}}
\CommentTok{// \{'a': [2,4], 'b': [1,2]\}}
\end{Highlighting}
\end{Shaded}

Notice that if we ignore backward assignment the interval for \texttt{b}
is \texttt{{[}0,4{]}}. Backward assignment lets us narrow down the
overapproximation, which means that we can have a tighter
overapproximation and consequently less false positives.

In this work, we make use of only non-relational Abstract Domains. A
relational Abstract Domain keeps some of the relationships between
variables, they are more expensive than non-relational Abstract Domains
but they allow even tighter overapproximations. See more
\textcite{mine_weakly_2004}.
