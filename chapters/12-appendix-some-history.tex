\chapter{Gotchas and Other ideas to Static Analysis in Python}%
\label{gotchas-and-other-ideas-to-static-analysis-in-python}

\ichange{This is just a DRAFT! REVISE!!}

\emph{\large Note: This chapter is probably the most informal of the whole
book. The principal idea of this chapter is to explain a little bit
deeper the history of the whole thesis, some of the things I tried to
solve the problem at hand, and what avenues I found could be explored
and which are probably dead ends. This whole chapter is me rambling
about various stuff}

In here, I write on the multiple other ways I thought on how to
Statically analyse the shape of tensors. I explored some of them but
never really finish them, while others just proved to be bad ideas
altogether.

Some background. The principal, original, idea for my master thesis was
to check the shape of tensors and check the validity of tensor
operations. I did not consider Abstract Interpretation as my first
option to statically analyse code. In fact my whole exposure to Static
Analysis had been Type Inference, some Data Flow Analysis algorithms,
and some exposure into SMT solvers theory and practice.

I will mention each one of the approaches I tried in chronological
order.

{\nonsection{Tensor Shape Type Checking - Haskell}}

Everything started in Haskell. I wrote a wrapper library around a
Haskell wrapper library for TensorFlow \autocite{abadi_tensorflow_2016}.
The library uses one of the lastest additions to the language, Dependent
Types \autocite{eisenberg_dependent_2016}. Dependent Types allow me to
define the shape of tensors at the type level, which allows the compiler
to check for the correctness of tensor operations at compilation time.

My main motivation to use Haskell is its very strict type system, and
the tools that it incorporates to work with types. For example, in GHC,
the standard Haskell implementation, one can ask for the inferred type
of an expression by annotating the type with the type symbol
\texttt{\_}\footnote{They are called holes. There are two types of
  holes: (regular) holes and type holes. A hole is an unknown expression
  and a type hole is an unknown type. You can put a whole at the type
  level and the compiler will tell you what is the type that should be
  at that position. I found this incredibly neat and useful!}, thus we
can ask what shape of tensor should we use in the middle of an
operation.

Although, type checking tensor shapes in Haskell proved to be possible
and not all too hard, albeit a very verbose at times, I yet have to know
the first Deep Learning enthusiast who writes in Haskell. So, I decided
to apply the same idea into Python, the \enquote{language} of Deep
Learning.

Some efforts have been made to add type inference and type checking into
Python (See Related Work, Chapter \ref{related-work})), the most
important one being MyPy.

{\nonsection{Extending MyPy}}

MyPy builds on the idea of Gradual Typing. MyPy is able to infer the
type of primitive values and allows optional annotations to check the
code. Sadly, MyPy does not consider literal numbers as valid types
(\texttt{int}, \texttt{float}, \texttt{List{[}int{]}} are valid types
but no \texttt{3} or \texttt{List{[}2,\ int{]}}), so it is currently
impossible to type annotate a variable with the size of the variable,
e.g.~something like \texttt{np.ndarray{[}2,\ 3,\ 4{]}}.

But hope is still not lost! We can encode numbers in MyPy's Type System
in a similar way to how was done in
\autocites{chen_typesafe_2017}{eaton_statically_2006}. We can encode 1
as \texttt{Tuple{[}None{]}}, 2 as \texttt{Tuple{[}None,\ None{]}}, and
so on.

Additionally, we can compare fields between two variables in an
operation with the use of \texttt{TypeVar}s (variables at the type
level). With these two ideas, we can rudimentary check the shapes of
tensors. First we write a stub file \autocite{pep484} with type
annotations to check for the shape of NumPy arrays:

\begin{Shaded}
\begin{Highlighting}[]
\CommentTok{# numpy.pyi}
\ImportTok{from}\NormalTok{ typing }\ImportTok{import}\NormalTok{ Any, Generic, Optional, Tuple, TypeVar}

\NormalTok{S }\OperatorTok{=}\NormalTok{ TypeVar(}\StringTok{'S'}\NormalTok{)  }\CommentTok{# Used for dtype}
\NormalTok{Shape }\OperatorTok{=}\NormalTok{ TypeVar(}\StringTok{'Shape'}\NormalTok{)  }\CommentTok{# Used for the shape}
\NormalTok{s1 }\OperatorTok{=}\NormalTok{ TypeVar(}\StringTok{'s1'}\NormalTok{)  }\CommentTok{# Another variable to use as reference to a shape}
\NormalTok{s2 }\OperatorTok{=}\NormalTok{ TypeVar(}\StringTok{'s2'}\NormalTok{)}
\NormalTok{s3 }\OperatorTok{=}\NormalTok{ TypeVar(}\StringTok{'s3'}\NormalTok{)}


\KeywordTok{class}\NormalTok{ ndarray(Generic[S, Shape]):}
    \KeywordTok{def} \FunctionTok{__init__}\NormalTok{(}\VariableTok{self}\NormalTok{, shape: Any) }\OperatorTok{->} \VariableTok{None}\NormalTok{: ...}

    \KeywordTok{def}\NormalTok{ dot(}
            \VariableTok{self}\NormalTok{: }\StringTok{'ndarray[S,Tuple[s1,s2]]'}\NormalTok{,}
\NormalTok{            b: }\StringTok{'ndarray[S,Tuple[s2,s3]]'}
\NormalTok{    ) }\OperatorTok{->} \StringTok{'ndarray[S,Tuple[s1,s3]]'}\NormalTok{: ...}

    \KeywordTok{def} \FunctionTok{__add__}\NormalTok{(}
            \VariableTok{self}\NormalTok{: }\StringTok{'ndarray[S,Shape]'}\NormalTok{,}
\NormalTok{            value: }\StringTok{'ndarray[S,Shape]'}
\NormalTok{    ) }\OperatorTok{->} \StringTok{'ndarray[S,Shape]'}\NormalTok{: ...}


\KeywordTok{def}\NormalTok{ array(}\BuiltInTok{object}\NormalTok{: Any) }\OperatorTok{->}\NormalTok{ ndarray[Any, Shape]: ...}
\end{Highlighting}
\end{Shaded}

And now we write the piece of code to type check:

\begin{Shaded}
\begin{Highlighting}[]
\ImportTok{import}\NormalTok{ numpy }\ImportTok{as}\NormalTok{ np}
\ImportTok{from}\NormalTok{ typing }\ImportTok{import}\NormalTok{ Tuple}

\NormalTok{one   }\OperatorTok{=}\NormalTok{ Tuple[}\VariableTok{None}\NormalTok{]}
\NormalTok{two   }\OperatorTok{=}\NormalTok{ Tuple[}\VariableTok{None}\NormalTok{, }\VariableTok{None}\NormalTok{]}
\NormalTok{three }\OperatorTok{=}\NormalTok{ Tuple[}\VariableTok{None}\NormalTok{, }\VariableTok{None}\NormalTok{, }\VariableTok{None}\NormalTok{]}
\NormalTok{four  }\OperatorTok{=}\NormalTok{ Tuple[}\VariableTok{None}\NormalTok{, }\VariableTok{None}\NormalTok{, }\VariableTok{None}\NormalTok{, }\VariableTok{None}\NormalTok{]}

\CommentTok{# x has shape (2,3)}
\NormalTok{x }\OperatorTok{=}\NormalTok{ np.array([[}\DecValTok{1}\NormalTok{, }\DecValTok{2}\NormalTok{, }\DecValTok{3}\NormalTok{], [}\DecValTok{4}\NormalTok{, }\DecValTok{5}\NormalTok{, }\DecValTok{6}\NormalTok{]])  }\CommentTok{# type: np.ndarray[float, Tuple[two, three]]}

\CommentTok{# y has shape (4,1)}
\NormalTok{y }\OperatorTok{=}\NormalTok{ np.array([[}\DecValTok{7}\NormalTok{], [}\DecValTok{0}\NormalTok{], [}\DecValTok{2}\NormalTok{], [}\DecValTok{1}\NormalTok{]])    }\CommentTok{# type: np.ndarray[float, Tuple[four, one]]}

\NormalTok{w }\OperatorTok{=}\NormalTok{ x }\OperatorTok{+}\NormalTok{ y  }\CommentTok{# fails! And MyPy warns us!! :D}

\CommentTok{# THIS SHOULD FAIL IN MYPY!! But it doesn't :(}
\NormalTok{z }\OperatorTok{=}\NormalTok{ x.dot(y)  }\CommentTok{# type: np.ndarray[float, Tuple[two, one]]}
\end{Highlighting}
\end{Shaded}

Everything seems to work perfectly, MyPy detects that the two numpy
arrays cannot be added together because the have different shapes, but
MyPy does not detect the error on multiplying two non-compatible
matrices. We told MyPy in the stub that the shape of the last dimension
in \texttt{self} (\texttt{s2}) should be the same as the first dimension
in \texttt{b}, but MyPy does not verify this condition
(\texttt{three\ !=\ four}).

Unfortunatelly, MyPy's \texttt{TypeVar}s are not fully-fledged type
variables. MyPy does not handle Dependent Types, so a solution as the
one suggested for Haskell could not completely be coded.

Two options came from this experiment, either look how to extend MyPy
type variables handling to work in the general case or ignore type
checking as a possible solution at the moment.

{\nonsection{StyPy}}

As I mentioned in Chapter \ref{pytropos-analysis-implementation}, before
I encountered Abstract Interpretation as a way to Statically Analyse
code\footnote{I know, it is a very old, well understood, and broadly
  applied analysis technique but I didn't know much about the broad
  world of Static Analysis before I delved deep into it.} I found
\emph{StyPy} \autocite{ortin_towards_2015}.

The idea behind StyPy is to reuse the infraestructure of the language to
analyse to not write everything from scratch. The idea is to take the
code to analyse and rewrite/transform it into a new piece of code that
when run it performs Static Type Analysis. This approach presents to
difficulties: first, it was planned as a technique for Static Type
Analysis not Static Value Analysis, and second, it is a very naïve and
novel idea, it is not very well formalised.

When you require to know only the type of a variable, its value becomes
irrelevant. If we do not consider any weird introspection Python
capabilities, it does not matter how many times you run a loop a
variable will always keep the same type after the first loop iteration,
i.e.~there is no need to consider complex scenarios and joining
operations for the type of a variable as we need to consider for the
value of a variable inside a loop.

There have not been any updates on the theory behind the project since
the paper was published in 2015.

All in all, the idea of reusing some of the infraestructure by rewriting
the code and running it in the same platform where one wants to analyse
the code is a very interesting idea. In fact, this idea has been applied
in other static analyses, for example, \textcite{lauko_symbolic_2018}
explain how they built a symbolic analyser for LLVM bytecode by reusing
some of the infraestructure of LLVM and not rewriting a complete
interpreter for LLVM from the ground up.

Pytropos started from the same idea of StyPy, transform and run modified
code. The idea would be that any variable in the (extended) interpreter
could be any of Python's or \texttt{Any} (\texttt{?}, the special type
introduced in Gradual Typing). An \texttt{Any} value would tell me that
I have no idea of the value contained in the variable, just that the
variable contains a variable. I discarded the idea of modifying StyPy as
it would mean rewriting the whole program for my purpose. So I went on
looking different projects that I could modify to implement the Static
Value Analysis I was considering.

{\nonsection{Modify PyPy}}

Once I considered building a special purpose interpreter to analyse the
shape of tensors (as an Abstract Interpreter is), I took a look in other
projects that had implemented interpreters for python. a very
interesting project is pypy. pypy \autocite{krekel_pypy_2007} is a
project focused on providing an alternative implementation of python
with added characteristics. most importantly, pypy is written in python
(or a subset of python, called rpython) and pypy can be extended/its
semantics can be modified. apparently, pypy offers a way to change the
semantics of the interpreter\footnote{the object space can be modified
  as explained in \url{http://doc.pypy.org/en/latest/objspace.html}}.

both characteristics; pypy being written in python, allowing for an easy
introspection and modification of its internals, and the ease of
modifing some of its semantics; made me think of using pypy as a
\enquote{base} for the custom made interpreter. i did not follow this
route because the work to be done to allow having different states of a
program and joining them (used in \texttt{if} branching when choice can
be made between two branches) seemed to require rewritting a big chunk
of the core of pypy. starting from zero seemed a better idea.

{\nonsection{abstract interpretation}}

in the end, i decided to build my own interpreter from scratch. it took
me a couple of months to arrive to something relatively usable but i
found out in the process that branching, loops, exceptions, and breaking
control statements made the implementation of a interpreter that used
\texttt{any} values very challenging (how are the semantics of a
non-deterministic \texttt{if}, of a loop, an exception, etc).

it became clear that trying to build an (extended) interpreter from
scratch would be very hard if no clear formal framework was used to put
everything into perspective. enter abstract interpretation. i cannot say
that i found abstract interpretation as the solution for the many
problems that an extended value \enquote{system} presented, because i
had no idea such thing existed.

as you may have noticed, i just arrived a the \enquote{proposed}
solution of my problem (static value analysis) by \enquote{mere} chance.
i was looking at the literature of static analysis, but i thought there
was little i could use from it. how wrong was i! always so
smartass{\todo{change wording}}.

anyway, once i studied together with my advisors the theory of abstract
interpretation, i knew that ai was the solution for my problem. it was a
long path nontheless, as it required to formalise a couple of things
about the language and define an abstract domain for the state of the
language, which proved to be challenging but rewarding.

now, i must mention here that i did came across a project that was doing
pretty much what i wanted to do, namely an abstract interpreter for
python. {\todo{add ref to lyra}} lyra is a project led by {\todo{fill
me}} which purpose is to build an abstract interpreter for python. there
are two reasons why i chosed not to use lyra for this thesis: the first
is purely egothistical and not practical at all, i wanted to build
something from my own, something from \enquote{scratch}\footnote{please,
  dear advisors, do not kill me for writing this!}; the second reason
was that i had found the idea of reusing some of the python
infraestructure too fascinating to let it into oblivion, rewriting
python code to run it into with python seems just a nice metaidea.

{\nonsection{Other Ideas}}

Other routes, I did considered to follow but did not because of time
restrictions where:

\begin{itemize}
\tightlist
\item
  If all we wanted was to check the shape of tensors and tensor
  operations for a couple of scenarios, we could build a Static Analysis
  out of various Data Flow Algorithms. Data flow analyses are simple,
  well studied, and already implemented in a couple of Static Analysis
  frameworks {\todo{add reference to an Overview of Program Analysis
  framework that the mention (I think it was for Netbeans or some other
  IDE)}}. The only work would be combining them adecuately to know the
  shape of tensors and consequently check for the correctness of a
  couple of cases.
\item
  Another option may have been encoding the shapes as constraints. The
  constraints could be checked using an out of the shelve equation
  solver (SMTs). In fact, a big chunk of the work for this has already
  been done, an implementation for Python for example {\todo{Add
  reference on ETH people doing this}}.
\end{itemize}
