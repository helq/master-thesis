\chapter{Introduction}\label{introduction}

\section{Motivation}\label{motivation}

Dynamically typed programming languages (dynamically typed languages for short) like
Python and Ruby have risen in popularity over the last two decades. Their simple syntax,
ease of use, out-of-the-box REPLs\footnote{Read-eval-print loop}, dynamic type systems,
and their great number of libraries have been the main factors to this rise. Indeed,
developers can  write, execute and see results immediately in fast development cycles,
especially at the initial stages. However, their simple syntax and dynamic typing make
them particularly difficult to analyse as the more rigid the language, the easier to
analyse it is. Most efforts on static analysis for Python have focused on assuming some
restrictions on the language. For example, MyPy \autocite{lehtosalo2016mypy} assumes the
type of variables as static.

Dynamically typed languages only check type mismatches at runtime, i.e.~errors between
variable types only get detected when the code is run. For example,
consider the following piece of Python code:

\begin{pythoncode*}{gobble=2}
  myvar = "Some text in a string"
  other = 4.1
  result = "oh my" + 2  # This will fail
  print(result + other, myvar)
\end{pythoncode*}

The code is syntactically correct, but it fails to execute. Python will interpret the first
two lines and will stop in the third, where it will throw an exception because strings
and numbers cannot be added together in Python.

A Static Type Analysis tool, such as MyPy \autocite{lehtosalo2016mypy}, is able to detect
type mismatches without ever executing the code.  MyPy has the ability to infer the type
of variables, e.g.~it detects in the piece of code from above that \pycode/other/ is of type
\pycode/float/ and \pycode/myvar/ of type \pycode/str/.

Python dynamic nature makes it extraordinarily hard to statically analyse. In fact,
due to Python introspection capabilities, it is impossible to guarantee the behaviour of a
piece of Python code without assuming some fixed basic semantics by the interpreter.  MyPy
tackles the undecidability problem of Python dynamic nature by restricting the number
of valid Python programs to those that conform to a static type system\footnotemark.

\footnotetext{I would like to express my immense gratitude to the MyPy team for writing
  such an amazing tool. I do not know how many hours they saved me from endless
  frustration at debugging. All Python code I write now is full of typing annotations to
  allow MyPy to check for any inconistency. What I like about MyPy is that it alerts me of
  faulty code as I write. It feels remarkably similar to writing code in a language with a
  stronger static type system (e.g.~Haskell).}

The inference power of MyPy is limited in several ways, one of them being it impossible to
infer the type of a function. MyPy allows developers to optionally add type annotations
into the code. Type annotations allow more restrictive and precise types, and they help
MyPy to catch more potential bugs.

Although MyPy is capable of checking for type mismatches, its assumptions limit it
from a whole class of valid and correct Python code\footnotemark. There are many scenarios
where even with type annotations, MyPy cannot detect a bug in the code. For example,
consider the following piece of python code:%\footnotemark:

\footnotetext{For example, there is no way (currently) to make MyPy happy about the
  function \pycode/def addtwice(x, y): return x + y + y; addtwice(3,2); addtwice('a','b')/.
  If no type annotation is given, MyPy alerts the user on the usage of non-typed
  functions. But no type can satisfy the assertion because of the \enquote{polymorphisity}
  of the \pycode/+/ operation.}

%\footnotetext{Notice the lack of type annotations in this example. MyPy is able to infer the
%  type of the variables in this example.}

\begin{pythoncode*}{gobble=2}
  import numpy as np
  x = np.array( [[1,2,3], [4,5,6]] ) # shape (2,3)
  y = np.array( [[7], [0], [2], [1]] ) # shape (4,1)
  z = np.dot( x, y ) # trying to apply dot product
\end{pythoncode*}

In this example, we are using NumPy \autocite{oliphant2006guide}, which is a well known
Python library, but it could also have been TensorFlow \autocite{abadi_tensorflow_2016} or
any other library with support for tensors and tensor operators. Tensors are a
generalisation of vectors and arrays, a homogeneous container which can be indexed by one
natural number or a list of natural numbers. As an example consider a tensor of only one
dimension, an array of size \texttt{n}, which contains exactly \(n\) elements. A matrix of
size \(n \times m\), alternatively, a tensor with shape \texttt{(n,\ m)}, is a structure
that holds \(n \cdot m\) elements. Notice that a shape is any tuple of natural numbers and
a tensor is a structure with a shape. For example, the shape \texttt{(2,\ 5,\ 6)}
tells us that the tensor has \(2 \cdot 5 \cdot 6 = 60\) elements.

The example above will fail to run as it hits an erroneous condition at the last line. Two
arrays/tensors are created, \texttt{x} and \texttt{y} with shapes \texttt{(2,\ 3)} and
\texttt{(4,\ 1)}, respectively. Then, a matrix multiplication with both tensors tries to
be performed, but it fails because the shapes of the two matrices are incompatible
(\(3 \ne 4\)). The call to \texttt{np.dot} fails, but the developer will not be warned about
it and will only notice it when executing the code. To summarise, the aforementioned piece
of code type checks in MyPy even though it fails to run.

If it were possible to add the shape of the tensors as part of the type definitions, we
could potentially type check for mismatches of tensor shapes. Unfortunately, there does
not seem to be a straightforward way to add the shape of tensors to their types and check
them in MyPy.

Tensors allow writing computations more concisely. Tensor operations are also highly
optimisable and can be run in parallel. With the advent of Deep Learning, tensors have
become a central figure in the developers' toolkit. Tensor operations fail if the tensors
are not of the right shape. To statically analyse if operations between tensors are valid,
the shape of tensors must be computed. To compute the shape of tensors, we require to
also compute the value of every other variable in the language. The kind of static
analysis that is focused on finding out the value of variables is called \textbf{Static
Value Analysis}.

\section{Problem Definition}\label{problem-definition}

Tensors are becoming an essential abstraction in the toolkit of developers. Tools tailored
to work with or based on tensors have been popping out in recent years. It is therefore
imperative to develop analysis tools to verify and flag potential bugs using tensors and
tensor operations. A Static Value Analysis tool focused on tensor shape analysis could aid
developers in their work.

All of the theoretical approaches that tackle the problem focus on type checking tensors
and tensor operations. \textcite{griffioen_type_2015},
\textcite{slepak_array-oriented_2014}, and \textcite{rink_modeling_2018} define a type
system extended with tensors and tensor operations. The principal idea of these type
systems is to encode the restrictions that the operations on tensors require. A tensor
operation is valid if the restriction checks with the tensors.

Practical attempts to check for tensor shapes have been mainly focused on languages with strong
type systems. \textcite{chen_typesafe_2017} (in Scala) and
\textcite{eaton_statically_2006} (in Haskell) have tried to annotate tensors' types with
their shapes, leaving the compiler's type check inference system to check the shapes. At
the time \textcite{eaton_statically_2006} proposed his methodology to extend types with
additional data, GHC (Glasgow Haskell Compiler), the by default Haskell compiler, didn't
have all the capabilities to work with data at the type level, resulting in some rather
complicated code to read and write. A recent approach to type check tensors in Haskell has
been shown in a library written by \textcite{elkin_haskell_2018}, which uses the updated
Dependent Type System of Haskell to code and enforce type shapes.

Abstract Interpretation is a framework for static analysis. The main idea of Abstract
Interpretation is to overapproximate the result of computing a piece of code. Any piece of
code can be run in an Abstract Interpreter, an interpreter build from the semantics
defined by Abstract Interpretation, in a finite amount of time. To warranty that the
Abstract Interpreter runs in a finite amount of time, Abstract Interpretation defines a
series of rules to follow in order to overapproximate the result of executing operations by the
program. Abstract Interpretation is especially useful for Static Value Analysis as the
rules to derive the semantics of an Abstract Interpreter from the formal semantics of the
language are very well studied, formalised and explicit \autocite{cousot_abstract_1977}.

Writing an Abstract Interpreter for Python requires the formal semantics of Python as a
starting point. Unfortunately, Python does not have any official formal semantics defined.
Some attempts to define formal semantics for Python have been done
\autocites{politz_python_2013}{fromherz_static_2018}{guth_formal_2013}{ranson_semantics_2008}
but none of them takes into account type annotations, which we hope to integrate into the
Static Value Analysis.

\textbf{Problem:} Building an Abstract Interpreter for Python to Statically Value-Analyse
tensor operations.

%{\inlinetodo{Should I add some research questions?}}

\subsubsection*{Assumptions and Scope}

In this section, we will explore some of the assumptions and scope of the Abstract
Interpreter to be build in the rest of the document.

Very little can be asserted from a piece of code in a programming language like Python
without making any assumptions about the environment. For example, consider the following
piece of code:

\begin{pythoncode*}{gobble=2}
  # runme.py
  a = int('6') + 2
  print(7)
  assert a == 8, "int('6') + 2 != 8. Wut?"
\end{pythoncode*}

One may be tempted to say that the code above never fails, but one would be mistaken.  It
will never fail to run if we assume that the program starts from a blank state, i.e. no
variables have been defined before its execution. In fact, the functions \pycode+int+ and
\pycode+print+ can be redefined to compute anything we want. Also, thanks to the
introspection capabilities of Python, the code above could do just about anything, and
that is without considering that the piece of code could be evaluated by an \pycode+exec+
or an \pycode+eval+ statement. For example, consider the following piece of code that
calls \texttt{runme.py}:

\begin{pythoncode}
  runme = open('runme.py').read()
  fakerun = """
  def int(val):
    return 3.0
  def print(val):
    global a
    a = val
  """ + runme
  exec(fakerun)
\end{pythoncode}

From now on, we will assume that any piece of code written in Python will be called from
the interpreter directly without changing the behaviour of any built-in function or
variable (this includes the behaviour of variables and functions from non-built-in
libraries like NumPy), i.e.~we assume that the code will be run from a blank state with no
global variables (re)defined.

As a developer, I can say that I like when I see a piece of code flagged wrong and IT IS
WRONG and not a false alarm, a false positive. It is very annoying to have a tool flag
every sentence you write as wrong even though the code works just fine. We will assume
from here onwards that we want the Tensor Analysis to flag only errors that are going to
happen if the code is run, i.e.~we want only true positives not false positives.

Although TensorFlow \autocite{abadi_tensorflow_2016}, PyTorch
\autocite{paszke2017pytorch} and other libraries would benefit from the development of a
tool targeted to check them, we chose to focus on checking NumPy's tensor operations.
NumPy is the standard library for computing with tensors and it is used as back-end on
most tensor projects.

Implementing an Abstract Interpreter, or any Static Analysis, for a language like Python
is a considerable undertaking, mainly because of the breadth of characteristics a mature
Programming Language like Python has. Thus, we will centre just on a couple of Python core
characteristics and will leave others for future work. The characteristics explored in
this current document are:

\begin{itemize}
\tightlist
\item Built-in primitive variables: \pycode|int|, \pycode|float|, \pycode|bool|,
  \pycode|None|, \pycode|list|, \pycode|tuple|.
\item Primitive functions: \pycode|int|, \pycode|print|, \pycode|input|,
  \ldots{}
\item Boolean and numeric operations: \pycode|+|, \pycode|-|, \pycode|*|, \pycode|/|,
  \pycode|\%|, \pycode|**|, \pycode+<<+, \pycode|>>|, \pycode|//|, \pycode+<+,
  \pycode|<=|, \pycode+>+ and \pycode+>=+.
\item \pycode|if| and \pycode|while| statements.
\item Import statement (limited only to the \pycode|numpy| library).
\item NumPy arrays and some of its methods/functions to work with NumPy arrays (including
  \pycode|dot|, \pycode|zeros|, \pycode|shape|, and numeric operations with broadcasting).
\end{itemize}

Consider the following piece of code:

\begin{pythoncode}
if someval:
  i = 3
else:
  i = "ntr"
print(i)
\end{pythoncode}

The variable \pycode|i| can either be an \pycode|int| or a \pycode|str|, thus the type of
\pycode|i| should be \pycode|Union[int,str]|.

Gradual Typing restricts the number of valid programs to those that type check. A variable
in Gradual Typing is of a specific type and it never changes, i.e.~if the type of a
variable is set to be \pycode+int+ then it can only contain \pycode+int+s. \pycode+Union+
types are meant to indicate that a variable may take more than one value at any point in
time.

Building an Abstract Interpreter aware of \pycode|Union| types is not considered part of
this work because if of the added complexity it would require. Therefore if a variable
holds more than one value, e.g.~it is either a \pycode|5| or \pycode|2.3|, then the type
of the variable will be considered to be \pycode|Any|, not \pycode|Union|. \pycode|Any|
and \pycode|Union| types are defined in the \pycode|typing| library \autocite{pep484}, an
implementation of Gradual Typing\footnote{%
  Gradual Typing is one kind of Static Type Analysis targeted to Dynamically Typed Languages
  like Python}.

It may seem like an Abstract Interpreter where variables can only carry one value at the
time would make the analysis weaker than Gradual Typing, but it does not make it necessarily.
Consider the following piece of code:

\begin{pythoncode}
i = 3
i = "ntr"
i += "ueor"
print(i)  # the type of `i` is `str`
\end{pythoncode}

Under Gradual Typing, the type of \pycode+i+ is \pycode+Union[int,str]+, but we know that
the variable \pycode+i+ holds only one type at \textbf{all} times, i.e. \pycode+i+ never
holds two or more values as in the previous example. The Abstract Interpreter would run
each line sequentially, and it would never find a state where the value of \pycode+i+ is
both \pycode+int+ and \pycode+str+ at the same time, i.e.~the variable is never considered
to be of type \pycode+Any+ but it will have type \pycode+int+ and then type \pycode+str+
as the code is run. In this regard, an Abstract Interpreter can give a better
approximation of what the types of a variable are at any point in time, while Gradual
Typing considers all possible types a variable may have at any point in time.

To recapitulate, the assumptions and scope of the abstract interpreter are:

\begin{itemize}
\tightlist
\item The behaviour of built-ins, imported variables and functions is always the same,
  and it is determined by the Python reference manual or the library's author,
  e.g.~\pycode+input+ is a function that returns a \pycode+str+.
\item The user will only be reported of errors that will happen but no errors that
  \emph{may} happen (This means that the resulting tool is not a verification tool).
\item The Static Value Analysis should focuse on checking operations from the NumPy library.
\item Only a selected subset of Python characteristics and functions are explored.
\item A variable can hold only one type of value at the time. If a variable is set to hold
  values of different types at the same time, then its value will be \pycode+Any+.
\end{itemize}

\section{Objectives}\label{objectives}

\subsection{General objective}\label{general-objective}

To design and implement a strategy for statically analyse tensor shapes in Python to
support early detection of potential bugs before execution.

\subsection{Specific objectives}\label{specific-objectives}

\begin{enumerate}
\def\labelenumi{\arabic{enumi}.}
\tightlist
\item To design and implement an abstract interpreter for Python.
\item To design and implement a strategy that uses the abstract interpreter to analyse the
  shapes of tensors for Python code.
\item To implement a tool that flags the bugs inferred by the abstract interpreter.
\item To empirically evaluate the developed static analysis in a set of representative
  test cases.
\end{enumerate}

\section{Contributions}\label{contributions}

The contributions of this work are the following:

\begin{itemize}
\tightlist
\item An specification of operational semantics for a subset of Python using a methodology
  similar to \textcite{fromherz_static_2018}. This specification includes a subset of the
  Python syntax, an AST representation of the syntax more malleable to work with, and the
  semantics of the subset of the Python Language.
\item Definition of an Abstract Domain for variables in Python.
\item Definition of an aliasing-aware Abstract Domain for the state of a Python program.
\item Derivation of an operational semantics from the Abstract Domain, i.e.~an Abstract
  Interpreter for Python.
\item A working, open source implementation of the Abstract Interpreter nicknamed
  Pytropos\footnotemark{} with a plugin for (Neo)Vim to work as a linter.
\end{itemize}

\footnotetext{The source code for the interpreter can be found at
  \url{https://github.com/helq/pytropos}}

\section{Thesis Structure}\label{thesis-structure}

The remaining parts of this document are organised as follows:

\begin{itemize}
\tightlist
\item Chapter~\ref{background} explores some background material on Dynamically typed and
  Statically typed languages, Tensors (and why are they important), the NumPy library, and
  Abstract Interpretation;
\item Chapter~\ref{pytropos-analysis-implementation} presents some of the gory details of
  the implementation of the Abstract Interpreter;
\item Chapter~\ref{validation-and-discussion} focuses on the tests made to ensure that the
  implementation follows the Abstract Interpreter nicknamed Pytropos, with the goal of
  showcasing the abilities of Pytropos to detect errors using NumPy arrays and to
  point some of the areas of improvement of the tool;
\item Chapter~\ref{related-work} presents some related work on Abstract Interpretation for
  Python, Static Analysis of Tensor shapes and related libraries, Static Analysis tools;
  and finally,
\item Chapter~\ref{conclusions} sums up the work done and future directions.
\item Additionally, an Appendix~\ref{appendix-ai-theory} to explain in detail the parts of
  the abstract interpreter. The appendix is divided into three parts: the definition of
  the syntax and semantics of a subset of Python, the definition of an Abstract
  Interpreter for Python, and some other ways in which Statically Analyse the shape of
  tensors for Python.
\end{itemize}
